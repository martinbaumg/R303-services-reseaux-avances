\documentclass[12pt, a4paper]{article}
\usepackage[francais]{babel}
\usepackage{pgfplots}
\usepackage{caption}
\usepackage{graphicx}
\usepackage[T1]{fontenc}
\usepackage{listings}
\usepackage{geometry}
\usepackage{minted}
\usepackage{array,multirow,makecell}
\usepackage[colorlinks=true,linkcolor=black,anchorcolor=black,citecolor=black,filecolor=black,menucolor=black,runcolor=black,urlcolor=black]{hyperref}
\setcellgapes{1pt}
\makegapedcells
\usepackage{fancyhdr}
\pagestyle{fancy}
\lhead{}
\rhead{}
\chead{}
\rfoot{\thepage}
\lfoot{Martin Baumgaertner}
\cfoot{}
\renewcommand{\footrulewidth}{0.4pt}
\renewcommand{\headrulewidth}{0.4pt}
\renewcommand{\listingscaption}{Code}
\renewcommand{\listoflistingscaption}{Table des codes}
% \usepackage{mathpazo} --> Police à utiliser lors de rapports plus sérieux

\begin{document}
\begin{titlepage}
	\newcommand{\HRule}{\rule{\linewidth}{0.5mm}} 
	\center 
	\textsc{\LARGE iut de colmar}\\[1.5cm] 
	\textsc{\Large R303}\\[0.5cm] 
	\textsc{\large Année 2022-23}\\[0.5cm]
	\HRule\\[0.75cm]
	{\huge\bfseries Services réseaux avancés}\\[0.4cm]
	\HRule\\[1.5cm]
	\textsc{\large martin baumgaertner}\\[0.5cm] 

	\vfill\vfill\vfill
	{\large\today} 
	\vfill
\end{titlepage}
\newpage
\tableofcontents
\newpage
\section{CM 1 - 6 septembre 2022}
\subsection{Rappels}
TCP : permet d'avoir une communication fiable de bout en bout, on a une session 
qui permet de faire du recouvrement, s'il y a un paquet perdu on le retransmet, pour ça on 
utilise un numéro de séquence, et on a un mécanisme d'aquittement qui nous dit quel
octet est arrivé et quel octet est perdu. \\

UDP : est utilisé lorsque l'accusé de réception des données n'a aucune signification, 
est un bon protocole pour les données circulant dans une seule direction,
est simple et adapté aux communications basées sur des requêtes,
n'est pas orienté connexion,
ne fournit pas de mécanisme de contrôle de congestion.\\

\begin{figure}[H]
    \centering
    \includegraphics[width=0.9\textwidth]{img/osi.png}
    \caption{Le modèle OSI}
    \label{fig:osi}
\end{figure}
\newpage
\subsection{Les DNS}
Le Domain Name System (DNS) est une base de données distribuée organisée de manière
hiérarchique, un protocole applicatif qui permet à un hôte de l'intérrroger. Le
port utilisé par défaut pour DNS est le port 53. L'objectif de ce sytème est
d'identifier les hôtes grâce à un nom. \\
Le système DNS est souvent utilisé en amont des protocoles applicatifs comme 
HTTP ou les mails.\\

\begin{figure}
\begin{tikzpicture}
    \begin{axis}[	grid= major ,
            width=0.8\textwidth ,
            xlabel = {Année} ,
            ylabel = {Nombre (en milliard)} ,
            xmin = 1991, xmax = 2022,
            ymin = 0, ymax = 7 000 000 000,
            legend entries={Nombre site web, Nombre d'internautes},
            legend style={at={(0,1)},anchor=north west}]
        \addplot coordinates {(1991,1) (2001,29 000 000) (2011,346 000 000) (2022,2 000 000 000) }; % Tracé point à point
        \addplot coordinates {(2000,414 000 000) (2005,1 000 000 000) (2011,2 000 000 000) (2022,5 000 000 000) }; % Tracé point à point
       
    \end{axis}
\end{tikzpicture}
\caption{Nombre de site web et d'internautes en fonction des années}
\end{figure}


\end{document}